\documentclass[10pt,a4paper]{moderncv}
\moderncvtheme[blue]{classic}                
\usepackage[utf8]{inputenc}
\usepackage[scale=0.8]{geometry}

\firstname{Nicolas}
\familyname{Morey-Chaisemartin}
\title{Architecte logiciel embarqué - Développeur Front-End et Système d'exploitation}
\address{8 Rue Tristan Corbière}{38400 Saint Martin d'Hères}    
\mobile{+33 6 78 31 51 62}                    
\email{nicolas@morey-chaisemartin.com}                      
\homepage{www.nicolas.morey-chaisemartin.com}
\extrainfo{29/06/1986}

\begin{document}
\maketitle
\section{Expérience professionnelle}
\cventry{Juillet 2009\\à Aujourd'hui}{Architecte et développeur logiciel embarqué}{Kalray}{Montbonnot}{}
{
\begin{itemize}
\item[-]{Développement de système d'exploitation et librarie runtime pour l'embarqué.}
\item[-]{Développement d'un \textbf{compilateur} front-end pour un langage de Streaming hautement parallèle.}
\item[-]{Responsable de l'\textbf{intégration continue} sous Git/Hudson.}
\item[-]{Co-administrateur IT.}
\item[-]{Environnement: \textbf{Systèmes embarqués}, MPPA, C, Assembleur, Ruby, Programmation parallèle.}
\end{itemize}
}

\cventry{Octobre 2008\\à Juillet 2009}{Ingénieur en Systèmes embarqués}{Coframi}{Meylan}{}
{
\begin{itemize}
\item[-]{Mission chez Bull dans l'équipe R\&D HPC: Expert \textbf{Infiniband}.}
\item[-]{Développement d'\textbf{algorithmes de routage} fault-tolerant(dépot de brevets).}
\item[-]{Mise en place de solution de monitoring réseau pour grand clusters.}
\item[-]{Environnement: Infiniband, Routage, QoS, \textbf{Drivers Linux}, \textbf{HPC}.}
\end{itemize}
}

\cventry{Mai 2008\\à Septembre 2008}{Projet de fin d'études et de Master}{Northumbria University}{Newcastle}{UK}
{
\begin{itemize}
\item[-]{Design et implémentation d'un \textbf{OS temps-réel} pour le Cell Broadband Engine.}
\item[-]{Environnement: \textbf{Multicore}, Temps-réel, \textbf{Scheduling}, Synchronisation, C, Assembleur.}
\end{itemize}
}

\cventry{Juin 2007\\à Aôut 2007}{Stagiaire IT}{Allibert-Trekking}{Montmélian}{}
{
\begin{itemize}
\item[-]{Support utilisateur et serveur sur un réseau Max OS X.}
\item[-]{Mise en place d'un \textbf{système de monitoring}.}
\item[-]{Développement d'outils de synchronisation entres les outils d'inventaire, de monitoring et d'administration.}
\item[-]{Environnement: Mac OS X, Php, Perl, Python, Bash, MySQL, openDirectory.}
\end{itemize}
}

\cventry{Janvier 2007}{Intervenant Junior-Entreprise}{MVA Limited}{Grenoble}{}
{
\begin{itemize}
\item[-]{Création d'un outil d'extraction de coordonnées GPS depuis Google Maps.}
\item[-]{Environnement: \textbf{Java}, Swing}
\end{itemize}
}

\cventry{Octobre 2006}{Intervenant Junior-Entreprise}{Institut Laue-Langevin}{Grenoble}{}
{
\begin{itemize}
\item[-]{Portage d'une application de démonstration de cristallographie d'objective-C vers Java.}
\item[-]{Environnement: Java, Swing, HTML.}
\end{itemize}
}

\cventry{Aôut 2003}{Stagiaire}{Xyalis}{Grenoble}{}
{
\begin{itemize}
\item[-]{Réalisation d'un \textbf {programme de non régression} pour toute la gamme des outils Xyalis.}
\item[-]{Environnement: Bash, Tcl, Tk.}
\end{itemize}
}

\cventry{Juillet 2001}{Stagiaire}{Xyalis}{Grenoble}{}
{
\begin{itemize}
\item[-]{Développement de scripts de démonstrations d'outils de CAO.}
\item[-]{Intégration d'outils dans une même GUI.}
\item[-]{Démonstration des logiciels de l'entreprise à un salon internation de micro-électronique aux USA (DAC)}
\item[-]{Environnement: Tcl, Tk.}
\end{itemize}
}

\section{Diplômes et Études}
\cventry{2010}{Ingénieur informatique}{ENSIMAG}{Grenoble}{}{}
\cventry{2010}{Master en Systèmes Embarqués}{Université de Northumbria}{Newcastle}{UK}{
\begin{itemize}
\item[]{En parallèle de la dernière année à l'ENSIMAG.}
\item[]{Année réalisée à Newcastle - Angleterre.}
\end{itemize}
}

\section{Expérience Personnelle}
\cventry{2008 \\à Aujourd'hui}{Participation à des projets Open Source}{}{}{}
{
\begin{itemize}
\item[-]{Participations au mailing list at au développement de \textbf{Linux}, \textbf{Git}, OpenSM.}
\end{itemize}
}

\cventry{2011 \\à Aujourd'hui}{Damage (DAtabase MetA GEnerator)}{}{}{}
{
\begin{itemize}
\item[-]{Développement d'un ensemble de script de génération de format intermédiaire.}
\item[-]{Permet à partir d'une description YAML, de générer toutes les méthodes de lecture/écriture/modification vers des formats XML/YAML/binaire depuis du C, Java ou Ruby.}
\item[-]{Environnement: Ruby, GPL}
\end{itemize}
}
\cventry{2006}{Participation au concours IBM sur le Cell Broadband Engine}{}{}{}
{
\begin{itemize}
\item[-]{Développement d'un framework open-source pour les algorithmes de type $producteur/consommateur$ sur un cluster de Cell.}
\item[-]{Environnement: C++, sockets}
\end{itemize}
}
\cventry{2005 \\à 2007}{Responsable informatique N'Sigma, Junior Entreprise de l'ENSIMAG}{}{}{}
{
\begin{itemize}
\item[-]{Administration d'un réseau mixte Linux/Windows.}
\item[-]{Participation au chiffrage des contrats N'Sigma.}
\end{itemize}
}

\cventry{2005}{Réalisation d'un micro-processeur}{}{}{}
{
\begin{itemize}
\item[-]{Développement en VHDL d'un micro-processeur 8bits avec carte vidéo sur FPGA.}
\end{itemize}
}

\cventry{2004}{Concours national de programmation Prologin}{}{}{}
{
\begin{itemize}
\item[]{Développement d'une IA pour un jeu de stratégie.}
\item[]{Classé \textbf{5ème}}
\end{itemize}
}

\section{Compétences}
\subsection{Informatique}
\cvcomputer
{Langages}{\textbf{C}, \textbf{Ruby}, Java, Assembleur, Tcl, Ada, Php, C++}
{Outils de développement}{Emacs, Netbeans, SQLDevelopper, Eclipse}
\cvcomputer
{Systèmes d'exploitation}{\textbf{Kernel}, \textbf{Scheduling}, Programmation concurrentielle, \textbf{Drivers}, Signaux}
{Gestionnaires de versions}{\textbf{Git}, \textbf{Gitolite}, \textbf{Hudson}, CVS, SVN, SCCS}


\cvcomputer
{Administration}{\textbf{Linux}, FreeBSD, Samba, Postfix, Apache, Shell scripting (Bash, Sed, Awk)}
{Réseau}{Routage, Firewall, TCP/IP, Ethernet, \textbf{Sockets}, QoS, \textbf{Infiniband}}

\cvcomputer
{Environnements graphiques}{\textbf{X11}, Gtk, Tk, Swing}
{Bases de données}{\textbf{Conception}, Optimisation, \textbf{SQL}, Oracle, MySQL, LDAP.}

\cvcomputer
{Conception Logicielle}{\textbf{UML}}
{Conception Electronique}{\textbf{VHDL}, Modelsim}

\cvcomputer
{Outils mathématiques}{Matlab, CoinOR, GPLK}{}{}

\subsection{Langues}
\cvlanguage{Anglais}{lu, parlé, écrit}{
Score au TOEFL:100/120 - TOEIC: 935/990
}
\cvlanguage{Allemand}{notions}{}

\section{Divers}
\cvline{Loisirs}{Randonnée à pied et à ski, photographie, lecture, cinéma, voyages.}
\cvline{Sport}{Ancien pilote deltaplance, membre de l'équipe de France Espoir.}

\end{document}
